\textbf{Notify Me: Observations, Conclusions and Experiences} \footnote{Aveiga Iván, Enríquez Wilson, Sornoza Andrés.}


\section{Description}
NotifyMe is an app that provides a service of reminders, the new feature is the of use GPS to detect when the user is near the place were he needs to realize the task.

\section{Observations}

As observations we highlight the use of GoogleMaps' API on mobile apps, and the usage of mobile development on Android. \\

The use of Versioning is very helpful while developing, because it offers advantages such as: history of commits, workingroups, reverse commits and contributions of every member in the team. We used Github as  our Versioning Software.

\section{Conclusiones}
It was very beneficial to work with GoogleMaps's API because the big data of GoogleMaps provided a rich experience. GoogleMaps \\ offers a better maps service than Bing, Nokia, and Apple.

We have analized the possibility to work with Google Place to implement categories for future versions of NotifyMe. \\

\section{Experience}

Working on a project for Android was a very educational experience. Although we were using Java as the primary language for the project, there were a lot of new things we had to learn. For example, working with the layout of the application and emulating a cellphone to test the application we made. \\

We had some setbacks, like the database for the app, and a small problem with the map where it didn’t show the saved locations. But there was tons of information available on the internet to solve those problems. This was one of the reasons why I liked working on Android, the amount of information and tutorials are huge, which makes it a lot easier to learn.\\

In the end I feel learning this was a very useful experience, because of the standings of the current market and the demand for Android apps. 
